% !Mode:: "TeX:UTF-8" 

%====================================== 中文摘要 ==========================================
\newpage\thispagestyle{empty}
%\BiAppendixChapter{摘~~~~要}{ABSTRACT (Chinese)}
%\setcounter{page}{1}\pagenumbering{Roman}
\begin{center}{\bfseries\xiaoer\hei {摘~~~~要}}\end{center}
\defaultfont

边界元法与有限元法相比,具有输入数据少、降低问题维数、所占计算机内存少、计算省时、计算精度高及可求解连续场点等优点;尤其对于无限域问题它是严密成立的。因此,对于大量实际问题,尤其是大区域岩土工程问题,边界元法比有限元法更具优越性。

在导师等前人已研究、开发的FSM•DDM间接边界元法(IBEM)数值计算系统的基础上,进一步完善了三维弹性问题的3D-FSM•DDM边界元数值系统;开发了多介质3D-FSM•DDM边界元数值子系统;研究并开发了考虑节理滑移和张开情况下的3D-FSM•DDM耦合的IBEM分析子系统;研究了考虑地形及构造应力影响下,基于3D-FSM间接边界元法的初始地应力场的反演分析方法,并开发了相应数值子系统;针对3D-FSM•DDM数值系统主程序,初步开发了配套的3D-FSM•DDM边界元前、后处理子系统。

上述3D-FSM•DDM间接边界元数值系统经开发完善后将会形成具有独立知识产权的大型三维边界元应用软件,以期对相关研究及应用做出贡献。
关键词:间接边界元法(IBEM);应力不连续法(FSM);位移不连续法(DDM);节理岩体;多介质

%\footnote{本研究得到某某基金 (编号:) 资助。}
{\bfseries\xiaosi\hei{关键词:}}间接边界元法(IBEM);应力不连续法(FSM);位移不连续法(DDM);节理岩体;多介质


\clearpage

%====================================== 英文摘要 ==========================================
\newpage\thispagestyle{empty}
\begin{center}{\bfseries\xiaoer {Abstract}}\end{center}
\defaultfont

Compared with the Finite Element Method (FEM), the Boundary Element Method (BEM) has many advantages, such as decreasing data input quantity, reducing the dimensions of the question, occupying the fewer computer memory, saving the time of computation, having the higher computation precision and solving continuous field point. It is established strictly to the infinite field question. So the BEM has the superiority to many actual problems, especially for the far field Rock and Soil engineering projects.

This article has consummated the elastic 3D-FSM•DDM IBEM numerical simulation system more completely based on the results which is researched and developed by the predecessors. And it also has researched and developed several 3D-FSM•DDM numerical simulation subsystem, i.e. the multi-medium 3D-FSM•DDM numerical simulation subsystem; the 3D-FSM•DDM coupling IBEM analysis subsystem with considering the joint slip and separation; the initial ground stress back analysis method and usable program based on the 3D-FSM IBEM with considering the influence of terrain and tectonic stress; the 3D-FSM•DDM BEM preprocessing subsystem and post-processing subsystem adapting to the master program of 3D-FSM•DDM IBEM numerical system. 

When the 3D-FSM•DDM IBEM numerical simulation system is consummated, it will be a large-scale application software of 3D-BEM with the independent intellectual property rights. It will make the contribution to the correlation research and the application.

{\bfseries\hei {Keywords:}} Indirect boundary element method(IBEM); Fictitious stress method(FSM); Displacement discontinuity method(DDM); Joint element; Multi-medium

\clearpage

