% !Mode:: "TeX:UTF-8" 

%====================================== 图清单 ==========================================
\newpage\thispagestyle{empty}
%\setcounter{page}{1}\pagenumbering{Roman}
%\BiChapter{图清单}{List of Figures}
%\BiAppendixChapter{图清单}{List of Figures}
\begin{center}\setcounter{page}{1}\pagenumbering{Roman}\end{center}
\begin{center}{\hei\fontsize{18pt}{18pt}\selectfont\setlength{\parskip}{0.5\baselineskip} 图清单}\end{center}   %黑体小2居中,单倍行距,段前0.5行,段后0.5行
\addcontentsline{toc}{chapter}{\protect\fontsize{12pt}{12pt}\selectfont\vspace{-0.9\baselineskip}\noindent 图清单}
\addcontentsline{toe}{chapter}{\protect\fontsize{12pt}{12pt}\selectfont\vspace{-0.4\baselineskip}\noindent List of Figures}
 %黑体小4号,单倍行距,段前断后0
 %\end{raggedright} 
{\fontsize{10.5pt}{16pt}
\newcolumntype{C}[1]{>{\centering\arraybackslash}p{#1}}
\centering
\begin{tabularx}{\textwidth}{|C{0.15\textwidth}|C{0.618\textwidth}|C{0.15\textwidth}|}
\hline
图序号 & 图名称 & 页码 \\
\hline
图2.1 & 边界元法与有限元法、有限差分比较 & 43  \\
\hline
Fig.2.1 & Difference among FDM, FEM \& BEM & 43  \\
\hline
图2.2 & 边界元法基本原理图示 & 43 \\
\hline
\end{tabularx}
%\caption{一个占满页宽的表格}
}% 中文宋体5号;英文:Times New Roman 5号字 



%====================================== 表清单 ==========================================

\newpage\thispagestyle{empty}
\begin{center}{\hei\fontsize{18pt}{18pt}\selectfont\setlength{\parskip}{0.5\baselineskip} {表清单}}\end{center}
\addcontentsline{toc}{chapter}{\protect\fontsize{12pt}{12pt}\selectfont\vspace{-0.9\baselineskip}\noindent 表清单}
\addcontentsline{toe}{chapter}{\protect\fontsize{12pt}{12pt}\selectfont\vspace{-0.4\baselineskip}\noindent List of Tables}
%黑体小4号,单倍行距,段前断后0
%\BiAppendixChapter{表清单}{List of Tables}
%\begin{center}{\bfseries\xiaoer\hei {表清单}}\end{center}

{\fontsize{10.5pt}{16pt}
\newcolumntype{C}[1]{>{\centering\arraybackslash}p{#1}}
\centering
\begin{tabularx}{\textwidth}{|C{0.15\textwidth}|C{0.618\textwidth}|C{0.15\textwidth}|}
\hline
图序号 & 图名称 & 页码 \\
\hline
图2.1 & 边界元法与有限元法、有限差分比较 & 43  \\
\hline
Fig.2.1 & Difference among FDM, FEM \& BEM & 43  \\
\hline
图2.2 & 边界元法基本原理图示 & 43 \\
\hline
\end{tabularx}
%\caption{一个占满页宽的表格}
}% 中文宋体5号;英文:Times New Roman 5号字 


%======================================变量注释 ==========================================
\newpage\thispagestyle{empty}
\begin{center}{\hei\fontsize{18pt}{18pt}\selectfont\setlength{\parskip}{0.5\baselineskip} {变量注释表}}\end{center}
\addcontentsline{toc}{chapter}{\protect\fontsize{12pt}{12pt}\selectfont\vspace{-0.5\baselineskip}\noindent 变量注释表}
\addcontentsline{toe}{chapter}{\protect\fontsize{12pt}{12pt}\selectfont\vspace{-0.1\baselineskip}\noindent List of Variables}

{\fontsize{10.5pt}{16pt}
\newcolumntype{C}[1]{>{\centering\arraybackslash}p{#1}}
\centering
\begin{tabularx}{\textwidth}{C{0.15\textwidth}C{0.618\textwidth}C{0.15\textwidth}}

\textbf{变量} & \textbf{注释} & \textbf{初现页} \\

E & 弹性模量,MPa & 43  \\

V & 泊松比 & 43  \\


\end{tabularx}
%\caption{一个占满页宽的表格}
}% 中文宋体5号;英文:Times New Roman 5号字 


 \clearpage

