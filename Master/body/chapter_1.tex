% !Mode:: "TeX:UTF-8" 

\pagenumbering{arabic}

\setcounter{page}{1} 
\BiChapter{绪\quad 论}{Introduction}
\begin{raggedright}{\bfseries\xiaoer\hei {1\hspace{0.5em}   Introduction}} %黑体小2加粗,单倍行距,段前0.5行,段后0行

\end{raggedright}
\setlength{\parskip}{0pt}
\defaultfont
%=========================================================================================
\BiSection{引 言}{Foreword}{(Foreword)}
  ………………………………………………………………………………………………………\par
   ………………………………………………………………………………………………………
\BiSubsection{线弹性力学的数值方法}

线弹性力学问题归结为在给定的边界条件下,求解一组线性偏微分方程组。在理论上,这种边值问题有唯一确定的解,但一般难以求得解析解。除弹性力学平面问题的复变函数解法属于正演解法外,其余弹性力学问题都只能用逆解法或半逆解法。逆解法和半逆解法的成功率很低,不能满足工程的需要。为此,人们在不断寻求解决问题的新途径。\par
 1 差分法\par
 以前,在得不到解析解的时候,人们或者采用差分法,按差分格式离散以获得数值解;或者按问题特点,选取试函数,采用里兹法或伽辽金法等近似方法来获得近似解。

  ………………………………………………………………………………………………………\par
   ………………………………………………………………………………………………………\par
   ………………………………………………………………………………………………………
%-----------------------------------------------------------------------------------------



